
\documentclass{article}
\usepackage[utf8]{inputenc}
\usepackage{graphicx}
\usepackage{xcolor}
\usepackage{geometry}
\usepackage{amsmath}
\geometry{a4paper, margin=1in}

\definecolor{addiuvaBlue}{HTML}{0050A0}
\definecolor{addiuvaPink}{HTML}{E20074}

\title{\textcolor{addiuvaBlue}{\textbf{Reporte Completo de Análisis de Datos ADS 2025}}}
\author{Transformación Empresarial}
\date{\today}

\begin{document}

\maketitle

\section{Introducción}
Este reporte presenta el análisis detallado de los servicios brindados en el periodo 2025. El objetivo es evaluar el cumplimiento de los Acuerdos de Nivel de Servicio (SLA) y la satisfacción del cliente (NPS).

\section{Fuentes de Información}
\begin{itemize}
    \item Dataset: \texttt{Servicios brindados ADS 2025 (1).xlsx}
    \item Hoja procesada: \texttt{BBDD}
\end{itemize}

\section{Metodología}
\subsection{Limpieza de Datos}
Se estandarizaron los nombres de columnas a formato \texttt{snake\_case}. Se procesaron fechas y horas vectorialmente para calcular la duración del servicio.

\subsection{Definiciones Conceptuales}
\begin{description}
    \item[SLA (Service Level Agreement)]: Tiempo máximo permitido entre la asignación y el contacto efectivo.
    \item[NPS (Net Promoter Score)]: Medida de lealtad del cliente.
\end{description}

\section{Fórmulas}
\subsection{Cálculo de SLA}
$$ Duraci\acute{o}n = Fecha_{Contacto} - Fecha_{Asignaci\acute{o}n} $$
$$ Estado = \begin{cases} 
CUMPLE & \text{si } Duraci\acute{o}n \le Umbral \\
NO CUMPLE & \text{si } Duraci\acute{o}n > Umbral 
\end{cases} $$
Umbrales: 45 min (Vial/Local), 90 min (Vial/Foráneo), 35 min (Legal/Local), 60 min (Legal/Foráneo).

\subsection{Cálculo de NPS}
$$ NPS = \% Promotores (9-10) - \% Detractores (0-6) $$

\section{Resultados Visuales}

\subsection{Cumplimiento SLA}
\begin{center}
    \includegraphics[width=0.7\textwidth]{../graficas/sla_pie.png}
\end{center}

\subsection{NPS Score}
\begin{center}
    \includegraphics[width=0.7\textwidth]{../graficas/nps_bar.png}
\end{center}

\subsection{Top Brokers}
\begin{center}
    \includegraphics[width=0.9\textwidth]{../graficas/top_brokers.png}
\end{center}

\section{Benchmarking con Boletín Oficial (Oct 2025)}
Se realizó una validación cruzada con los datos del Boletín de Calidad.

\begin{itemize}
    \item \textbf{SLA Oficial (Octubre)}: 85.59\%
    \item \textbf{SLA Calculado (Anual)}: 44.51\%
    \item \textbf{Varianza}: La diferencia se atribuye a la ventana de tiempo (Anual vs Octubre) y a posibles exclusiones manuales de "Justificaciones" no presentes en la metadata del Excel.
\end{itemize}

\section{Conclusiones}
El análisis revela oportunidades de mejora en la captura de tiempos (alta tasa de registros inválidos) y focalización en brokers con alto volumen pero bajo cumplimiento. Se recomienda auditar los casos con "Cancelado Posterior" para asegurar que no impacten los indicadores operativos reales.

\end{document}
